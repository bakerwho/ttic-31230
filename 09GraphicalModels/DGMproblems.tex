\documentclass{article}
\input ../preamble
\parindent = 0em
\parskip = 1ex

%\newcommand{\solution}[1]{}
\newcommand{\solution}[1]{\bigskip {\color{red} {\bf Solution}: #1}}

\begin{document}


\centerline{\bf TTIC 31230 Fundamentals of Deep Learning}

\bigskip

\centerline{\bf Problems for Graphical Models.}

\bigskip
\bigskip

{\bf Problem 1. Pseudolikelihood of a one dimensional spin glass.}
We let $\hat{x}$ be an assignment of a value to every node. We define the score of $\hat{x}$ by
$$f(\hat{x}) = \sum_{i = 1}^{N-1} \bbone[\hat{x}[i] = \hat{x}[i+1]]$$
The probability distribution over assignments is defined by a softmax.
$$Q_f(\hat{x}) = \softmax_{\hat{x}} f(\hat{x})$$
What is the {\bf Pseudoliklihood} of the all ones assignment?

\solution{
$$\tilde{P}_f(\hat{x}) = \Pi_i P_f(\hat{x}[i] \;|\; \hat{x}/i)$$

where $\hat{x}/i$ consists of all components of $\hat{x}$ other than $i$.  In a graphical model $P_f(\hat{x}[i] \;|\; \hat{x}/i)$ is determined by the neighbors of $i$ and we can consider only how a value is scored against it neighbors.  For $\hat{x}$ equal to all ones we have

\begin{eqnarray*}
f(\hat{x}) & = & N-1 \\
\\
f(\hat{x}[i = 0]) & = & \left\{\begin{array}{ll} N-3 & \mbox{for $1 < i < N$} \\ N-2 & \mbox{for $i=1$ or $i=N$} \end{array}\right.
\end{eqnarray*}
For $1 < i < N$ we get
\begin{eqnarray*}
Q_f(\hat{x}[i=1]\; |\;\hat{x}/i) & = & \frac{e^{N-1}}{e^{N-1} + e^{N-3}} \\
\\
& = & \frac{1}{1 + e^{-2}}
\end{eqnarray*}
and for $i = 1$ or $i = N$ we get
$$Q_f(\hat{x}[i=1]\; |\;\hat{x}/i) = \frac{1}{1 + e^{-1}}$$

This gives

$$\tilde{Q}(\hat{x}) = (1 + e^{-1})^{-2}(1 + e^{-2})^{-(N-2)}$$
}

\bigskip
{\bf Problem 2. Pseudolikelihood for images.} Consider a semantic segmentation $\hat{y}[i]$ on pixels $i$ with $\hat{y}[i]$
a semantic class label in $\{C_1,\ldots,C_K\}$.  We also assume a scoring function $s_\Phi$ on semantic segmentations defining
$$P_\Phi(\hat{y}) = \softmax_{\hat{y}}\;s_\Phi(\hat{y})$$
Pseudolikelihood is defined by
\begin{eqnarray*}
  \tilde{P}_\Phi(\hat{y}) & = & \prod_i P_\Phi(\hat{y}[i] \;|\; \hat{y}\backslash i)
\end{eqnarray*}
where $\hat{y}\backslash i$ assigns a class to every pixel other than $i$, and $\hat{y}[i:=c]$ is the semantic
segmentation identical to $\hat{y}$ except that pixel $i$ is labeled with semantic class $c$.  In a typical graphical model for images we have
$$P_\Phi(\hat{y}[i]\; |\; \hat{y}\backslash i) = P_\Phi(\hat{y}[i]\; | \;\hat{y}[N(i)])$$
where $\hat{y}[N(i)]$ is $\hat{y}$ restricted to those pixels which are neighbors of pixel $i$.

\medskip
(a) show
$$\frac{P_\Phi(\hat{y})}{\sum_c P_\Phi(\hat{y}[i:=c])} = \softmax_c s_\Phi(\hat{y}[i:=c])\;\;\;\mbox{evaluated at $c = y[i]$}$$

\solution{
\begin{eqnarray*}
  \frac{P_\Phi(\hat{y})}{\sum_c P_\Phi(\hat{y}[i:=c])} & = & \frac{\frac{1}{Z} e^{s_\Phi(\hat{y})}}{\sum_c \frac{1}{Z} e^{s_\Phi(\hat{y}[i:=c])}} \\
  \\
  & = & \frac{e^{s_\Phi(\hat{y})}}{\sum_c e^{s_\Phi(\hat{y}[i:=c])}} \\
  \\
  & = & (\softmax_c\;s_\Phi(\hat{y}[i:=c]))[\hat{y}[i]]
\end{eqnarray*}
}

\medskip
(b) How many scores need to be computed in the worst case for computing $P_\Phi(\hat{y})$.  Given the result of part (a), how many for computing $\tilde{P}_\Phi(\hat{y})$?

\solution{$K^N$ for $P_\Phi$ and $KN$ for $\tilde{P}_\Phi$.}

\medskip
(c) Consider a distribution on semantic segmentations where for each pixel the class assigned to that pixel is determined by the other pixels.
Can this distribution be defined by a softmax over scores?  Explain your answer.

\solution{
  No. Since $e^s > 0$ for any finite $s$, all semantic segmentations must have nonzero probability.
}

\medskip
(d) If $P_\Phi$ is a distribution defined in some other way such that the class of each pixel is completely determined by the other pixels,
given a simple expression for $\tilde{P}_\Phi(\hat{y})$ in the case where $\hat{y}$ has nonzero probability under $P_\Phi$.

\solution{We have $P_\Phi(\hat{y}|\hat{y}\backslash i) = 1$ which implies $\tilde{P}(\hat{y}) = 1$.}
  
\end{document}
